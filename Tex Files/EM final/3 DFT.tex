\section{傅立葉轉換與複數平面上的圓}
  \subsection{離散傅立葉轉換}
  複數域上的傅立葉轉換對於資訊相關的領域有極大的幫助,例如檔案壓縮與訊號處理等。不過在應用之前,我們還有最後一個數學上的問題:雖然我們為了使理論適用於非週期函數,而把傅立葉級數推廣為傅立葉積分,但是因為由機器執行的計算均為離散的,我們還是需要把積分置換回級數的形式作計算。
  \\\\
  在資訊領域中,我們通常會將傅立葉轉換另外定義為
  \[C(w)=\int_{-\infty}^{\infty}f(v)\exp{(-iwv)}\,dv\eqno{(2.1)}\]
  而要將積分式置換為級數式相當容易,我們只需要對原本的積分式取樣即可。舉例來說,若取\,\(N\)\,個式\,(2.1)\,中的\,\(f(v)\exp{(-iwv)}\)\,連加,我們即能得到\,\(C(w)\)\,的離散形式:
  \[C_a(w_k)=\sum_{n=0}^{N-1}f(v_n)\exp{(-iw_kv_n)}\,\Delta v\eqno{(2.2)}\]
  其中\,\(a\)\,代表了近似\,(Approximation)。
  \\\\
  在\,1.1\,的推導過程中,我們假設了
  \[w_n=\frac{n\pi}{L}\quad\text{與}\quad\Delta w=\frac{\pi}{L}\]
  兩個變數,而這兩個變數之間具有
  \[w_n=n\,\Delta w\]
  的關係。同樣地,我們可以假設
  \[w_k=k\,\Delta w\quad\text{與}\quad v_n=n\,\Delta v\]
  我們再定義
  \[\Delta w=\frac{2\pi}{N\Delta v}\quad\text{與}\quad X_k=\frac{C_a(w_k)}{\Delta v}\quad\text{與}\quad x_k=f(v_n)\]
  因此式\,(2.2)\,可以寫為
  \[X_k=\sum_{n=0}^{N-1}x_n\exp\left(-\frac{2\pi i}{N}kn\right)\eqno{(2.3)}\]
  我們亦可以由\,(1.17)\,的歐拉公式將上式表示為
  \[X_k=\sum_{n=0}^{N-1}x_n\left[\cos(2\pi kn/N)-i\sin(2\pi kn/N)\right]\eqno{(2.4)}\]
  而式\,(2.3)\,與式\,(2.4)\,便是離散傅立葉轉換\,(Discrete Fourier transform, DFT)\,的公式。
  \\\\
  仿照式\,(1.20)\,與式\,(2.3),我們便能將\,\(X_k\)\,轉換回\,\(x_n\):
  \[x_n=\frac{1}{N}\sum_{k=0}^{N-1}X_k\exp\left(\frac{2\pi i}{N}kn\right)\eqno{(2.5)}\]
  即
  \[x_n=\frac{1}{N}\sum_{k=0}^{N-1}X_k\left[\cos(2\pi kn/N)+i\sin(2\pi kn/N)\right]\eqno{(2.6)}\]
  以上兩式則稱為逆離散傅立葉轉換\,(Inverse discrete Fourier transform, IDFT)。

  \subsection{IDFT\,與本輪}
  因為複數平面以\,\(x\)\,軸作為實數軸,而以\,\(y\)\,軸作為虛數軸,故任一複數
  \[t\exp(i\theta)=t\left[\cos\theta+i\sin\theta\right]\]
  即表示了位於複數平面上的點\,\([t, \theta]\)。因此我們可以透過繪製一系列的圓,將代表函數的\,\(x_n\)\,計算出,而這些圓被稱為本輪\,(Epicycle):本輪的英文源自古希臘文\,ΕΠΙΚΥΚΛΟΣ,其義為運動於另一個圓圈之上的圓圈。它原先被早期的天文學家用作描述星體運動的模型,後來被數學家與電腦科學家作為複數域上傅立葉轉換的應用,尤其是\,IDFT。
  \\\\
  要繪製出這些圓,我們必須分別找出這些圓的其中三個、最具有代表性的物理量:
  \begin{enumerate}
    \item[(1)]
    半徑\,\(r\)、

    \item[(2)]
    頻率\,\(f\)、

    \item[(3)]
    相位角\,\(\phi\)。
  \end{enumerate}
  \noindent 由後兩項物理量,我們便可以建構出下一個圓所在的圓心位置。根據式\,(2.6),這三項物理量分別為
  \[r=\sqrt{\Re^2+\Im^2}\]
  \[f=k\]
  與
  \[\phi=\arctan(\Im/\Re)\]
  其中\,\(\Re\)\,與\,\(\Im\)\,分別代表了\,\(X_k/N\)\,的實部與虛部。
  \\\\
  半徑\,\(r\)\,與相位角\,\(\phi\)\,的定義都相對直觀,因為它們都直接表現在複數平面上,但頻率\,\(f\)\,就較為複雜:在一般的正弦函數
  \[f(t)=\sin(\omega t)\]
  中,\(\omega\)\,代表了這個函數的頻率,因此對於變數為\,\(n\)\,的式\,(2.6)\,而言,可以作為頻率的便是\,\(2\pi k/N\)。我們又捨棄對於級數的疊代來說為一常數的\,\(2\pi/N\),故\,\(f=k\)。
